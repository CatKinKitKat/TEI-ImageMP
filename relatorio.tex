\documentclass{article}

\usepackage{arxiv}

\usepackage[utf8]{inputenc} % allow utf-8 input
\usepackage[T1]{fontenc}    % use 8-bit T1 fonts´
\usepackage{hyperref}       % hyperlinks
\usepackage{url}            % simple URL typesetting
\usepackage{booktabs}       % professional-quality tables
\usepackage{amsfonts}       % blackboard math symbols
\usepackage{nicefrac}       % compact symbols for 1/2, etc.
\usepackage{microtype}      % microtypography
\usepackage{graphicx}
\usepackage{natbib}
\usepackage{doi}
\usepackage[portuguese]{babel}



\title{\textit{HIGH PERFORMANCE COMPUTING}}

%date{April 11, 2021}	% Here you can change the date presented in the paper title
%\date{} 				% Or removing it

\author{{\hspace{1mm}Bernardo Marques, 18373}\\
	Instituto Politécnico de Beja\\
	Engenharia Informática\\
	\And
	{\hspace{1mm}Gonçalo Amaro, 17440} \\
	Instituto Politécnico de Beja\\
	Engenharia Informática\\
	\And
	{\hspace{1mm}Hugo Silva, 16570} \\
	Instituto Politécnico de Beja\\
	Engenharia Informática\\
}

% Uncomment to remove the date
%\date{}

% Uncomment to override  the `A preprint' in the header
\renewcommand{\headeright}{Relatório da seccção de \textit{HPC} da disciplina de TEI}
\renewcommand{\undertitle}{Relatório da seccção de \textit{HPC} da disciplina de TEI}
\renewcommand{\shorttitle}{\textit{HPC}}

%%% Add PDF metadata to help others organize their library
%%% Once the PDF is generated, you can check the metadata with
%%% $ pdfinfo template.pdf
\hypersetup{
pdftitle={Relatório da seccção de HPC da disciplina de TEI},
pdfsubject={q-bio.NC, q-bio.QM},
pdfauthor={Bernardo Marques, Gonçalo Amaro, Hugo Silva},
pdfkeywords={HPC, TEI, Relatório},
}

\begin{document}
\maketitle

\begin{abstract}
Projecto que pretende demonstrar o uso da computação de alto desempenho, com um exemplo real e de um mini-projecto da detecção de contornos de \textit{Sobel} via programação \textit{multicore} em \textit{OpenMP}. 
\end{abstract}

\keywords{Contornos de \textit{Sobel} \and \textit{OpenMP}}

\section{Introdução}
O objectivo deste trabalho é clarificar o uso e o potencial da computação de elevado desempenho.

Este divide-se em duas partes: uma meramente explicativa sobre um projecto real com o uso desta categoria de computação: O Projecto \textit{Kaleidoscope} da \textit{Repsol} e, seguidamente, uma apresentação clara e detalhada do mini-projecto pedido: Detecção dos contornos de \textit{Sobel} numa imagem \textit{4K} (3840 x 2160 pixeis) com programação paralela.

\section{Projecto \textit{Kaleidoscope} da \textit{Repsol}}
\textit{Kaleidoscope} é um projecto de colaboração tecnológico entre a \textit{Emerson} e a \textit{Repsol} projectado para trazer tecnologias avançadas de imagem sísmica de sub-superfície para a indústria de petróleo e gás. Como parte do projecto, a \textit{Emerson} implementará e implantará soluções avançadas de imagem de sub-superfície com base nas tecnologias centrais da \textit{Repsol}. Combinando o que há de mais moderno em visualização computacional de ponta e em computação de alto desempenho, as soluções estarão disponíveis para a comunidade de geociências da \textit{Repsol} e para todas as empresas de petróleo e gás que optem por licenciar as tecnologias, para apoiar os seus processos de transformação digital.

\subsubsection{A máquina onde se processa}
A \textit{Repsol} aliou-se ao centro de super-computação de Barcelona, Espanha, a qual disponibiliza tempo-\textit{CPU} do seu supercomputador \textit{MareNostrum}.

O \textit{MareNostrum} é uma máquina com capacidade de processamento de 13.7 \textit{petaflops} e de armazenamento de 14 \textit{petabytes} e é o maior supercomputador da Península Ibérica.

Este divide-se um dois blocos: um de propósito geral e um de tecnologias emergentes.

O bloco principal tem 48 \textit{racks} com 3456 \textit{nodes} com um Lenovo ThinkSystem SD530, que é composto por dois Intel Xeon Platinum com 24 \textit{cores} (48 \textit{threads}) cada, um total de 165888 processadores e uma memória \textit{RAM} de 390 \textit{terabytes}.

O segundo bloco (de tecnologias emergentes) é composto de três \textit{clusters}, dos quais:
\begin{itemize}
	\item Um é composto por \textit{racks} de \textit{GPUs}, nomeadamente racks com \textit{NVIDIA} Volta \textit{GPUs} e outros com \textit{IMP POWER9}. Este \textit{cluster} tem uma capacidade computacional de 1.5 \textit{petaflops}.
	\item Outro composto por tecnologias da \textit{AMD}, com \textit{CPUs AMD EPYC} de arquitectura \textit{Rome} e \textit{CPUs Radeon Instinct MI50}. Este pequeno \textit{cluster} sozinho tem uma capacidade de 0.52 \textit{petaflops}, equivalente ao supercomputador \textit{Frontier} na \textit{Oak Ridge National Laboratory}.
	\item O ultimo é composto por processadores \textit{ARMv8} desenhados para alto desempenho, contrariamente aos \textit{ARMv8} que encontramos nos nossos dispositivos \textit{mobile} (que são desenhados para o menor custo energético e térmico). Com um poder total de 0.65 \textit{petaflops}.
\end{itemize}

\subsubsection{O porquê?}
Um \textit{survey} geológico a uma área geográfica, tem um \textit{dataset} de entre 10 a 20 \textit{terabytes}, uma pequena simulação de fluidos num computador \textit{desktop} comum, já requer algum tempo de renderização incluindo com paralelismo auxiliado pela \textit{GPU}, logo com um \textit{dataset} dessa larga escala, necessita de um outro nível de paralelismo e auxilio a algoritmos eficientes.

\subsection{Como funciona?}
É um sistema de simulação de fluidos altamente paralelizado com recurso a \textit{datasets} precisos e recentes de dados geológicos. Este divide a área de cálculo, quer o modelo seja bi ou tri-dimensional em células com percepção e comunicação entre si, onde os fluidos (cujo os mesmo são compostos de partículas) quando sua parte sai dos limites da célula e entra na célula adjacente é comunicado o seu estado e características (movimento, força, etc.).

\section{Projecto Algoritmo Deteção de \textit{Sobel} paralelizado}
O filtro \textit{Sobel} é uma operação utilizada no processamento de imagem, aplicado sobretudo em algoritmos de detecção de contornos.
Matematicamente este operador utiliza duas matrizes 3×3 que são convolvidas com a imagem original para calcular aproximações das derivadas - uma para as variações horizontais e uma para as verticais \cite{wikipedia}.

\begin{figure}[!htb]
\centering
\includegraphics[width=5cm]{x.png}
\caption{X - Direction Kernel}
\label{fig:x-kernel}
\end{figure}

\begin{figure}[!htb]
\centering
\includegraphics[width=5cm]{y.png}
\caption{Y - Direction Kernel}
\label{fig:Login}
\end{figure}



\subsection{A máquina onde se processa}
Lenovo Legion Y530
\begin{itemize}
	\item CPU: Intel i5-8300H (4 Cores e 8 \textit{Threads})
	\item GPU: Intel UHD 630 (+ NVIDIA 1050 4GB, porém desativada, no \textit{modeset})
	\item RAM: 16GB DDR4 2666MHz
	\item SSD: 256GB NVMe TLC
	\item HDD: 1TB SATA
\end{itemize}

\subsection{Como funciona?}
O Algoritmo de Deteção de Sobel calcula o gradiente da intensidade da imagem em cada pixel, dando a direcção da maior variação de claro para escuro e a quantidade de variação nessa direcção. Assim, obtém-se uma noção de como varia a luminosidade em cada pixel, de forma mais suave ou abrupta.

Com isto consegue-se estimar a presença de uma transição claro-escuro e de qual a orientação desta. Como as variações claro-escuro intensas correspondem a fronteiras bem definidas entre objectos, consegue-se fazer a detecção de contornos \cite{jasnau}.

\subsubsection{Como funciona? (O projecto)}

Para compilar os programas usamos o comando \textit{make}, que executa o \textit{makefile}.
Para executar os nossos executáveis, damos primeiro permissões de execução com o comando \textit{chmod +x nome\_do\_binario} e de seguida executamos com \textit{./nome\_do\_binario}.

Para cronometrar as execuções usamos o comando \textit{time ./nome\_do\_binario}.

\subsection{Diferença nos tempos de execução}
Graças ao uso de paralelismo recorrendo a biblioteca OpenMP, conseguimos acelerar o processamento da imagem, tal como podemos demonstrar nas cronometragem da execução dos binários.

\begin{figure}[!htb]
\centering
\includegraphics[width=15cm]{timings.png}
\caption{Diferença entre os timings}
\label{fig:timings}
\end{figure}

\newpage

\bibliographystyle{unsrtnat}
\begin{thebibliography}{9}
\bibitem{wikipedia} 
Sobel operator - Wikipedia, the free Encyclopedia
\textit{https://en.wikipedia.org/wiki/Sobel\_operator}.

\bibitem{jasnau} 
José Jasnau Caeiro
\textit{Aulas online da disciplina}. (2021)

\end{thebibliography}
\end{document}
